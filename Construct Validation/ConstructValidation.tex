% Options for packages loaded elsewhere
\PassOptionsToPackage{unicode}{hyperref}
\PassOptionsToPackage{hyphens}{url}
%
\documentclass[
  man]{apa6}
\usepackage{amsmath,amssymb}
\usepackage{lmodern}
\usepackage{iftex}
\ifPDFTeX
  \usepackage[T1]{fontenc}
  \usepackage[utf8]{inputenc}
  \usepackage{textcomp} % provide euro and other symbols
\else % if luatex or xetex
  \usepackage{unicode-math}
  \defaultfontfeatures{Scale=MatchLowercase}
  \defaultfontfeatures[\rmfamily]{Ligatures=TeX,Scale=1}
\fi
% Use upquote if available, for straight quotes in verbatim environments
\IfFileExists{upquote.sty}{\usepackage{upquote}}{}
\IfFileExists{microtype.sty}{% use microtype if available
  \usepackage[]{microtype}
  \UseMicrotypeSet[protrusion]{basicmath} % disable protrusion for tt fonts
}{}
\makeatletter
\@ifundefined{KOMAClassName}{% if non-KOMA class
  \IfFileExists{parskip.sty}{%
    \usepackage{parskip}
  }{% else
    \setlength{\parindent}{0pt}
    \setlength{\parskip}{6pt plus 2pt minus 1pt}}
}{% if KOMA class
  \KOMAoptions{parskip=half}}
\makeatother
\usepackage{xcolor}
\usepackage{graphicx}
\makeatletter
\def\maxwidth{\ifdim\Gin@nat@width>\linewidth\linewidth\else\Gin@nat@width\fi}
\def\maxheight{\ifdim\Gin@nat@height>\textheight\textheight\else\Gin@nat@height\fi}
\makeatother
% Scale images if necessary, so that they will not overflow the page
% margins by default, and it is still possible to overwrite the defaults
% using explicit options in \includegraphics[width, height, ...]{}
\setkeys{Gin}{width=\maxwidth,height=\maxheight,keepaspectratio}
% Set default figure placement to htbp
\makeatletter
\def\fps@figure{htbp}
\makeatother
\setlength{\emergencystretch}{3em} % prevent overfull lines
\providecommand{\tightlist}{%
  \setlength{\itemsep}{0pt}\setlength{\parskip}{0pt}}
\setcounter{secnumdepth}{-\maxdimen} % remove section numbering
% Make \paragraph and \subparagraph free-standing
\ifx\paragraph\undefined\else
  \let\oldparagraph\paragraph
  \renewcommand{\paragraph}[1]{\oldparagraph{#1}\mbox{}}
\fi
\ifx\subparagraph\undefined\else
  \let\oldsubparagraph\subparagraph
  \renewcommand{\subparagraph}[1]{\oldsubparagraph{#1}\mbox{}}
\fi
\newlength{\cslhangindent}
\setlength{\cslhangindent}{1.5em}
\newlength{\csllabelwidth}
\setlength{\csllabelwidth}{3em}
\newlength{\cslentryspacingunit} % times entry-spacing
\setlength{\cslentryspacingunit}{\parskip}
\newenvironment{CSLReferences}[2] % #1 hanging-ident, #2 entry spacing
 {% don't indent paragraphs
  \setlength{\parindent}{0pt}
  % turn on hanging indent if param 1 is 1
  \ifodd #1
  \let\oldpar\par
  \def\par{\hangindent=\cslhangindent\oldpar}
  \fi
  % set entry spacing
  \setlength{\parskip}{#2\cslentryspacingunit}
 }%
 {}
\usepackage{calc}
\newcommand{\CSLBlock}[1]{#1\hfill\break}
\newcommand{\CSLLeftMargin}[1]{\parbox[t]{\csllabelwidth}{#1}}
\newcommand{\CSLRightInline}[1]{\parbox[t]{\linewidth - \csllabelwidth}{#1}\break}
\newcommand{\CSLIndent}[1]{\hspace{\cslhangindent}#1}
\ifLuaTeX
\usepackage[bidi=basic]{babel}
\else
\usepackage[bidi=default]{babel}
\fi
\babelprovide[main,import]{english}
% get rid of language-specific shorthands (see #6817):
\let\LanguageShortHands\languageshorthands
\def\languageshorthands#1{}
% Manuscript styling
\usepackage{upgreek}
\captionsetup{font=singlespacing,justification=justified}

% Table formatting
\usepackage{longtable}
\usepackage{lscape}
% \usepackage[counterclockwise]{rotating}   % Landscape page setup for large tables
\usepackage{multirow}		% Table styling
\usepackage{tabularx}		% Control Column width
\usepackage[flushleft]{threeparttable}	% Allows for three part tables with a specified notes section
\usepackage{threeparttablex}            % Lets threeparttable work with longtable

% Create new environments so endfloat can handle them
% \newenvironment{ltable}
%   {\begin{landscape}\centering\begin{threeparttable}}
%   {\end{threeparttable}\end{landscape}}
\newenvironment{lltable}{\begin{landscape}\centering\begin{ThreePartTable}}{\end{ThreePartTable}\end{landscape}}

% Enables adjusting longtable caption width to table width
% Solution found at http://golatex.de/longtable-mit-caption-so-breit-wie-die-tabelle-t15767.html
\makeatletter
\newcommand\LastLTentrywidth{1em}
\newlength\longtablewidth
\setlength{\longtablewidth}{1in}
\newcommand{\getlongtablewidth}{\begingroup \ifcsname LT@\roman{LT@tables}\endcsname \global\longtablewidth=0pt \renewcommand{\LT@entry}[2]{\global\advance\longtablewidth by ##2\relax\gdef\LastLTentrywidth{##2}}\@nameuse{LT@\roman{LT@tables}} \fi \endgroup}

% \setlength{\parindent}{0.5in}
% \setlength{\parskip}{0pt plus 0pt minus 0pt}

% Overwrite redefinition of paragraph and subparagraph by the default LaTeX template
% See https://github.com/crsh/papaja/issues/292
\makeatletter
\renewcommand{\paragraph}{\@startsection{paragraph}{4}{\parindent}%
  {0\baselineskip \@plus 0.2ex \@minus 0.2ex}%
  {-1em}%
  {\normalfont\normalsize\bfseries\itshape\typesectitle}}

\renewcommand{\subparagraph}[1]{\@startsection{subparagraph}{5}{1em}%
  {0\baselineskip \@plus 0.2ex \@minus 0.2ex}%
  {-\z@\relax}%
  {\normalfont\normalsize\itshape\hspace{\parindent}{#1}\textit{\addperi}}{\relax}}
\makeatother

% \usepackage{etoolbox}
\makeatletter
\patchcmd{\HyOrg@maketitle}
  {\section{\normalfont\normalsize\abstractname}}
  {\section*{\normalfont\normalsize\abstractname}}
  {}{\typeout{Failed to patch abstract.}}
\patchcmd{\HyOrg@maketitle}
  {\section{\protect\normalfont{\@title}}}
  {\section*{\protect\normalfont{\@title}}}
  {}{\typeout{Failed to patch title.}}
\makeatother

\usepackage{xpatch}
\makeatletter
\xapptocmd\appendix
  {\xapptocmd\section
    {\addcontentsline{toc}{section}{\appendixname\ifoneappendix\else~\theappendix\fi\\: #1}}
    {}{\InnerPatchFailed}%
  }
{}{\PatchFailed}
\keywords{keywords\newline\indent Word count: X}
\DeclareDelayedFloatFlavor{ThreePartTable}{table}
\DeclareDelayedFloatFlavor{lltable}{table}
\DeclareDelayedFloatFlavor*{longtable}{table}
\makeatletter
\renewcommand{\efloat@iwrite}[1]{\immediate\expandafter\protected@write\csname efloat@post#1\endcsname{}}
\makeatother
\usepackage{lineno}

\linenumbers
\usepackage{csquotes}
\ifLuaTeX
  \usepackage{selnolig}  % disable illegal ligatures
\fi
\IfFileExists{bookmark.sty}{\usepackage{bookmark}}{\usepackage{hyperref}}
\IfFileExists{xurl.sty}{\usepackage{xurl}}{} % add URL line breaks if available
\urlstyle{same} % disable monospaced font for URLs
\hypersetup{
  pdftitle={Construct and Criterion-related validation of the Bifactor Engagement Scale},
  pdfauthor={Mike DeFabiis1, Casey Osorio2, Morgan Russell1, \& John Kulas3},
  pdflang={en-EN},
  pdfkeywords={keywords},
  hidelinks,
  pdfcreator={LaTeX via pandoc}}

\title{Construct and Criterion-related validation of the Bifactor Engagement Scale}
\author{Mike DeFabiis\textsuperscript{1}, Casey Osorio\textsuperscript{2}, Morgan Russell\textsuperscript{1}, \& John Kulas\textsuperscript{3}}
\date{}


\shorttitle{Bifactor Validation}

\authornote{

Add complete departmental affiliations for each author here. Each new line herein must be indented, like this line.

Enter author note here.

}

\affiliation{\vspace{0.5cm}\textsuperscript{1} Montclair State University\\\textsuperscript{2} Harver\\\textsuperscript{3} eRg}

\abstract{%
We present the results of a construct and criterion-related validation of a new measure of employee engagement.
}



\begin{document}
\maketitle

The roots of employee {[}aka work; e.g., W. Schaufeli and Bakker (2010){]} engagement research likely started with theoretical expansions of forms of employee participation (see, for example, Ferris \& Hellier, 1984) and job involvement (e.g., Elloy, Everett, \& Flynn, 1991). This exploration extended into broader considerations of attitudes and emotions (Staw, Sutton, \& Pelled, 1994) and were informed by further exploration of the dimensionality of constructs such as organizational commitment (Meyer \& Allen, 1991). The 1990's saw focused development and refinement {[}for example, a dissertation; Leone (1995) or actual semantic reference; William A. Kahn (1990a){]}. Staw et al. (1994) investigated the relationships between \emph{positive emotions} and favorable work outcomes, and although they do not use the word, ``engagement'', their distinction between felt and expressed emotion likely held influence upon the burgeoning interest in the engagement construct.

Clear in this history is the specification of engagement as a work \emph{attitude}.

Although occasionally referred to as residing on the opposing pole to \emph{burnout} (Christina Maslach \& Leiter, 2008), these two constructs are currently most commonly conceptualized as being distinct (Goering, Shimazu, Zhou, Wada, \& Sakai, 2017; Kim, Shin, \& Swanger, 2009; Wilmar B. Schaufeli, Taris, \& Van Rhenen, 2008; Timms, Brough, \& Graham, 2012), although certainly not universally (Cole, Walter, Bedeian, \& O'Boyle, 2012; Taris, Ybema, \& Beek, 2017). Comparing the two, Goering et al. (2017) concluded that they have a moderate (negative) association, but also distinct nomological networks. Wilmar B. Schaufeli et al. (2008) investigated both internal and external association indicators, concluding that engagement and burnout (as well as \emph{workaholism}) should be considered three distinct constructs.

Burnout can be defined as a psychological syndrome characterized by exhaustion (low energy), cynicism (low involvement), and inefficacy (low self-efficacy), which is experienced in response to chronic job stressors (e.g., Leiter \& Maslach, 2004; C. Maslach \& Leiter, 1997). Alternatively, engagement refers to an individual worker's involvement and satisfaction as well as enthusiasm for work (Harter, Schmidt, \& Hayes, 2002). W. B. Schaufeli and Bakker (2003) further specify a ``positive, fulfilling, work-related state of mind that is characterized by vigor, dedication, and absorption'' (p.~74). Via their conceptualization, vigor is described as high levels of energy and mental resilience while working. Dedication refers to being strongly involved in one's work and experiencing a sense of significance, enthusiasm, inspiration, pride, and challenge. Absorption is characterized by being fully concentrated and happily engrossed in one's work, whereby time passes quickly and one has difficulties with detaching oneself from work (Wilmar B. Schaufeli, Salanova, González-Romá, \& Bakker, 2002). The dimension of absorption has been noted as being influenced in conceptual specification by (Csikszentmihalyi, 1990)'s concept of ``flow''.

Regarding measurement, Gallup is widely acknowledged as an early pioneer in the measurement of the construct (see, for example, Coffman \& Harter, 1999). The Utrecht Work Engagement Scale (UWES) is another self-report questionnaire developed by W. B. Schaufeli and Bakker (2003) that directly assesses the vigor, dedication, and absorption elements.

\hypertarget{attitudes}{%
\subsection{Attitudes}\label{attitudes}}

\begin{quote}
TRIPARTITE MODEL--work here
\end{quote}

The first, to our knowledge, use of the word ``engagement'' as a construct came in William A. Kahn (1990b), defining it as: ``the harnessing of organization members' selves to their work roles; in engagement, people employ and express themselves physically, cognitively, and emotionally during role performances.'' Although this definition was quickly bypassed by subsequent papers (see, for example, (Baumruk, 2004) and (Shaw, 2005), who framed it in terms of one's cognitive and affective \emph{commitment} to one's organization), William A. Kahn (1990b)'s definition is notable in that it conforms to the then-ascendant tripartite model of attitudes proposed by Rosenberg (1960). This model frames attitudes as latent variables that manifest cognitively, affectively and behaviorally.

Although falling out of favor in the decades following its construction, interest in the tripartite model was revived by Kaiser and Wilson (2019),

The present article explores two methods for constructing a scale that incorporates both the substantive and attitudinal models into one, a more classical one based on corrected item-total correlations and one based on modification indices.

Existing measures include Soane et al. (2012)

Our conceptualization of work engagement is a mental state wherein employees: a) feel energized (\emph{Vigor}), b) are enthusiastic about the content of their work and the things they do (\emph{Dedication}), and c) are so immersed in their work activities that time seems compressed (\emph{Absorption}). We further decompose each of these facets into three attitudinal components: d) feeling (e.g., affect), e) thought (e.g., cognition), and f) action (e.g., behavior). Development and construct validation of the focal 18-item measure of engagement is described in Russell, Ossorio Duffoo, Garcia Prieto Palacios Roji, and Kulas (2022) whereas the current study on administrative response cues in the form of order of item presentation. The expectation is that either model (attitudinal or substantive) will exhibit stronger factorial validity when item administration parallels latent structure.

Our contributions:

\begin{enumerate}
\def\labelenumi{\arabic{enumi}.}
\tightlist
\item
  Methodological
\end{enumerate}

\begin{itemize}
\tightlist
\item
  Intentional bi-factor structure
\end{itemize}

\begin{enumerate}
\def\labelenumi{\arabic{enumi}.}
\setcounter{enumi}{1}
\tightlist
\item
  Practical
\end{enumerate}

\begin{itemize}
\tightlist
\item
  A new public domain measure of engagement
\item
  Scalable to two aggregations (research {[}DAC{]} and actionable {[}ABC{]})
\end{itemize}

\begin{enumerate}
\def\labelenumi{\arabic{enumi}.}
\setcounter{enumi}{2}
\tightlist
\item
  Theoretical
\end{enumerate}

\begin{itemize}
\tightlist
\item
  Possibly help explain some of the high inter-scale correlations reported with other measures
\end{itemize}

\hypertarget{methods}{%
\section{Methods}\label{methods}}

\hypertarget{participants}{%
\subsection{Participants}\label{participants}}

Of the 743 total Qualtrics panel respondents, 366 were excluded based on conservative indices of carelessness across the larger survey (consistent non-differentiating responses across more than 20 consecutive items or greater than 50\% missing responses. For Prolific panel respondents, 568 were retained of 785 total participants due to the same exclusion criteria. The smaller (\emph{n} = 232) snowball sample retained all participants for a total combined analysis sample of 1177.

\hypertarget{material}{%
\subsection{Material}\label{material}}

\hypertarget{procedure}{%
\subsection{Procedure}\label{procedure}}

A previous instrument administration reduced 36 candidate items to 20. Primarily for reason of equal balance, we wanted to ultimately land on 18 items (6 per attitudinal/substantive scale dimension, 2 per bifactor subscale). Two primary considerations were given to the decision to retain or delete the 6 deletion candidates: 1) is the content of the item necessary for the definitional content domain, and 2) does the empirical functioning of the item implicate possible revision/deletion. The items considered deletion candidates were from the Absorption-Cognition subscale (Item 1: \emph{I am able to concentrate on my work without getting distracted}, Item 3: \emph{Time passes quickly while I'm working}, and Item 4: \emph{I find it difficult to mentally disconnect from work}) and the Dedication-Cognition subscale (Item 25: \emph{I plan to stay with this company as my career advances}, Item 26: \emph{I believe this company cares about my career goals}, and Item 28: \emph{This organization challenges me to work at my full potential}).

\hypertarget{data-analysis}{%
\subsection{Data analysis}\label{data-analysis}}

We used R (Version 4.2.1; R Core Team, 2022) and the R-packages \emph{papaja} (Version 0.1.1; Aust \& Barth, 2022), and \emph{tinylabels} (Version 0.2.3; Barth, 2022) for all our analyses.

Looking first at the Absorption-Cognition candidate items, Item 4 stood out as a candidate for exclusion based on empirical indices (corrected item-total correlations, inter-item correlations, and bifactor analysis fit. Conceptually we also agreed that Item 4 was not uniquely critical for comprehensive coverage across either the Cognition or Absorption constructs. Figure presents the visual CFA.

\hypertarget{results}{%
\section{Results}\label{results}}

The final recommended scale definitions are located in Table.

\hypertarget{discussion}{%
\section{Discussion}\label{discussion}}

\newpage

\hypertarget{references}{%
\section{References}\label{references}}

\hypertarget{refs}{}
\begin{CSLReferences}{1}{0}
\leavevmode\vadjust pre{\hypertarget{ref-R-papaja}{}}%
Aust, F., \& Barth, M. (2022). \emph{{papaja}: {Prepare} reproducible {APA} journal articles with {R Markdown}}. Retrieved from \url{https://github.com/crsh/papaja}

\leavevmode\vadjust pre{\hypertarget{ref-R-tinylabels}{}}%
Barth, M. (2022). \emph{{tinylabels}: Lightweight variable labels}. Retrieved from \url{https://cran.r-project.org/package=tinylabels}

\leavevmode\vadjust pre{\hypertarget{ref-baumruk2004missing}{}}%
Baumruk, R. (2004). \emph{The missing link: The role of employee engagement in business success}. \emph{47}, 48--52.

\leavevmode\vadjust pre{\hypertarget{ref-coffman_hard_1999}{}}%
Coffman, C., \& Harter, J. (1999). A hard look at soft numbers. \emph{Position Paper, Gallup Organization}.

\leavevmode\vadjust pre{\hypertarget{ref-cole2012job}{}}%
Cole, M. S., Walter, F., Bedeian, A. G., \& O'Boyle, E. H. (2012). Job burnout and employee engagement: A meta-analytic examination of construct proliferation. \emph{Journal of Management}, \emph{38}(5), 1550--1581.

\leavevmode\vadjust pre{\hypertarget{ref-csikszentmihalyi1990flow}{}}%
Csikszentmihalyi, M. (1990). \emph{Flow: The psychology of optimal experience} (Vol. 1990). Harper \& Row New York.

\leavevmode\vadjust pre{\hypertarget{ref-elloy_examination_1991}{}}%
Elloy, D. F., Everett, J. E., \& Flynn, W. R. (1991). An examination of the correlates of job involvement. \emph{Group \& Organization Studies}, \emph{16}(2), 160--177. \url{https://doi.org/10.1177/105960119101600204}

\leavevmode\vadjust pre{\hypertarget{ref-ferris_added_1984}{}}%
Ferris, R., \& Hellier, P. (1984). Added value productivity schemes and employee participation. \emph{Asia Pacific Journal of Human Resources}, \emph{22}(4), 35--44. \url{https://doi.org/10.1177/103841118402200406}

\leavevmode\vadjust pre{\hypertarget{ref-goering2017not}{}}%
Goering, D. D., Shimazu, A., Zhou, F., Wada, T., \& Sakai, R. (2017). Not if, but how they differ: A meta-analytic test of the nomological networks of burnout and engagement. \emph{Burnout Research}, \emph{5}, 21--34.

\leavevmode\vadjust pre{\hypertarget{ref-harter_business-unit-level_2002}{}}%
Harter, J. K., Schmidt, F. L., \& Hayes, T. L. (2002). Business-unit-level relationship between employee satisfaction, employee engagement, and business outcomes: A meta-analysis. \emph{Journal of Applied Psychology}, \emph{87}(2), 268.

\leavevmode\vadjust pre{\hypertarget{ref-kahn1990psychological}{}}%
Kahn, William A. (1990b). Psychological conditions of personal engagement and disengagement at work. \emph{Academy of Management Journal}, \emph{33}(4), 692--724.

\leavevmode\vadjust pre{\hypertarget{ref-kahn_psychological_1990}{}}%
Kahn, William A. (1990a). Psychological conditions of personal engagement and disengagement at work. \emph{Academy of Management Journal}, \emph{33}(4), 692--724.

\leavevmode\vadjust pre{\hypertarget{ref-kaiser_campbell_2019}{}}%
Kaiser, F. G., \& Wilson, M. (2019). The {Campbell} {Paradigm} as a {Behavior}-{Predictive} {Reinterpretation} of the {Classical} {Tripartite} {Model} of {Attitudes}. \emph{European Psychologist}, \emph{24}(4), 359--374. \url{https://doi.org/10.1027/1016-9040/a000364}

\leavevmode\vadjust pre{\hypertarget{ref-kim_burnout_2009}{}}%
Kim, H. J., Shin, K. H., \& Swanger, N. (2009). Burnout and engagement: {A} comparative analysis using the {Big} {Five} personality dimensions. \emph{International Journal of Hospitality Management}, \emph{28}(1), 96--104. \url{https://doi.org/10.1016/j.ijhm.2008.06.001}

\leavevmode\vadjust pre{\hypertarget{ref-leiter_areas_2004}{}}%
Leiter, M., \& Maslach, C. (2004). Areas of worklife: A structured approach to organizational predictors of job burnout. In \emph{Research in occupational stress and well-being} (Vol. 3, pp. 91--134). \url{https://doi.org/10.1016/S1479-3555(03)03003-8}

\leavevmode\vadjust pre{\hypertarget{ref-leone_relation_1995}{}}%
Leone, D. R. (1995). \emph{The relation of work climate, higher order need satisfaction, need salience, and causality orientations to work engagement, psychological adjustment, and job satisfaction} (PhD thesis). ProQuest Information \& Learning.

\leavevmode\vadjust pre{\hypertarget{ref-maslach1997causes}{}}%
Maslach, C., \& Leiter, M. (1997). What causes burnout. \emph{Maslach C, Leiter MP. The Truth About Burnout: How Organizations Cause Personal Stress and What to Do about It. San Francisco, CA: Josey-Bass Publishers}, 38--60.

\leavevmode\vadjust pre{\hypertarget{ref-maslach_early_2008}{}}%
Maslach, Christina, \& Leiter, M. P. (2008). Early predictors of job burnout and engagement. \emph{Journal of Applied Psychology}, \emph{93}(3), 498--512.

\leavevmode\vadjust pre{\hypertarget{ref-meyer_three-component_1991}{}}%
Meyer, J. P., \& Allen, N. J. (1991). A three-component conceptualization of organizational commitment. \emph{Human Resource Management Review}, \emph{1}(1), 61--89.

\leavevmode\vadjust pre{\hypertarget{ref-R-base}{}}%
R Core Team. (2022). \emph{R: A language and environment for statistical computing}. Vienna, Austria: R Foundation for Statistical Computing. Retrieved from \url{https://www.R-project.org/}

\leavevmode\vadjust pre{\hypertarget{ref-rosenberg_cognitive_1960}{}}%
Rosenberg, M. J. (1960). Cognitive, affective, and behavioral components of attitudes. In \emph{Attitude organization and change}.

\leavevmode\vadjust pre{\hypertarget{ref-engage_2022}{}}%
Russell, M., Ossorio Duffoo, C., Garcia Prieto Palacios Roji, R., \& Kulas, J. (2022). Development of an intentional bifactor measure of engagement. \emph{The Seattle Edition of SIOP}, 1--14. SIOP.

\leavevmode\vadjust pre{\hypertarget{ref-schaufeli_uwesutrecht_2003}{}}%
Schaufeli, W. B., \& Bakker, A. B. (2003). {UWES}--utrecht work engagement scale: Test manual. \emph{Unpublished Manuscript: Department of Psychology, Utrecht University}, \emph{8}.

\leavevmode\vadjust pre{\hypertarget{ref-schaufeli_measurement_2002}{}}%
Schaufeli, Wilmar B., Salanova, M., González-Romá, V., \& Bakker, A. B. (2002). The measurement of engagement and burnout: A two sample confirmatory factor analytic approach. \emph{Journal of Happiness Studies}, \emph{3}(1), 71--92.

\leavevmode\vadjust pre{\hypertarget{ref-schaufeli2008workaholism}{}}%
Schaufeli, Wilmar B., Taris, T. W., \& Van Rhenen, W. (2008). Workaholism, burnout, and work engagement: Three of a kind or three different kinds of employee well-being? \emph{Applied Psychology}, \emph{57}(2), 173--203.

\leavevmode\vadjust pre{\hypertarget{ref-schaufeli_conceptualization_2010}{}}%
Schaufeli, W., \& Bakker, A. (2010). The conceptualization and measurement of work engagement. In W. Schaufeli, A. Bakker, \& M. Leiter (Eds.), \emph{Work engagement: A handbook of essential theory and research} (pp. 10--24). New York: Psychology Press.

\leavevmode\vadjust pre{\hypertarget{ref-shaw2005engagement}{}}%
Shaw, K. (2005). An engagement strategy process for communicators. \emph{Strategic Communication Management}, \emph{9}(3), 26.

\leavevmode\vadjust pre{\hypertarget{ref-soane2012development}{}}%
Soane, E., Truss, C., Alfes, K., Shantz, A., Rees, C., \& Gatenby, M. (2012). Development and application of a new measure of employee engagement: The ISA engagement scale. \emph{Human Resource Development International}, \emph{15}(5), 529--547.

\leavevmode\vadjust pre{\hypertarget{ref-staw_employee_1994}{}}%
Staw, B. M., Sutton, R. I., \& Pelled, L. H. (1994). Employee positive emotion and favorable outcomes at the workplace. \emph{Organization Science}, \emph{5}(1), 51--71.

\leavevmode\vadjust pre{\hypertarget{ref-taris2017burnout}{}}%
Taris, T. W., Ybema, J. F., \& Beek, I. van. (2017). Burnout and engagement: Identical twins or just close relatives? \emph{Burnout Research}, \emph{5}, 3--11.

\leavevmode\vadjust pre{\hypertarget{ref-timms2012burnt}{}}%
Timms, C., Brough, P., \& Graham, D. (2012). Burnt-out but engaged: The co-existence of psychological burnout and engagement. \emph{Journal of Educational Administration}, \emph{50}(3), 327--345.

\end{CSLReferences}


\end{document}
